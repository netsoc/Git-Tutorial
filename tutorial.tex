\documentclass[xcolor=dvipsnames]{beamer}
\usetheme{Rochester}
\usepackage{graphicx}
\usecolortheme[named=OliveGreen]{structure}
%\useoutertheme{infolines}
\useinnertheme{rounded}
\setbeamertemplate{blocks}[rounded][shadow=false]
\setbeamertemplate{items}[ball]
\begin{document}
\title[Netsoc Git Tutorial] {Dublin University Internet Society}
\author[A. Anderson \& C. Vize]{Andrew Anderson \inst{1} \and Colm Vize \inst{2}}
\institute[TCD]{\inst{1} School Of Computer Science and Statistics \and \inst{2} Tophat Software}
\subtitle{Git Tutorial}{An introduction to the git VCS}
\date{\today}
\titlepage

\begin{frame}
	\frametitle{What is a VCS}
	
	Stands for Version Control System.\\
	A system used to control collaborative projects, so that multiple people can work on it at the same
	time without conflicts (as far as possible).\\
	CVSes normally work by layering "commits" on top of each other. A commit is a set of changes from the previous state,
	that can be used to produce the new state.\\
	Git is a DVCS, or Distributed CVS. This means that every user has a full copy of the project history.\\
\end{frame}

\begin{frame}
	\frametitle{A tutorial introduction}

	Lets start by creating a repo
	\begin{block}{Create a new repo}
		mkdir myrepo\\
		cd myrepo\\
		git init
	\end{block}
\end{frame}

\begin{frame}
	\frametitle{Cloning repositories}
	Makes a local copy of a repository, from e.g github\\
	Works with lot of protocols (http, ssh, native git protocol, even local files).

	\begin{block}{Clone a local repository}
		git clone /path/to/repo
	\end{block}
	\begin{block}{Clone a remote repository (These slides!)}
		git clone https://github.com/netsoc/Git-Tutorial.git
	\end{block}
\end{frame}

\begin{frame}
	\frametitle{Make your changes}
	After the cloning a repo, as on the previous slides, you will have a folder with the repositories
	files in it (normally source code for a program).

	Make your changes, and then...
\end{frame}

\begin{frame}
	\frametitle{Add and commit}
	Commit them.
	You can stage chages (select them for committing) by using
	\begin{block}{}
		git add $\langle$filename$\rangle$ \# adds $\langle$filename$\rangle$\\
		git add . \# also works on directories
	\end{block}
	To actually commit these changes use
	\begin{block}{}
		git commit -m "Commit message"
	\end{block}
	commits the changes we have added, with the message "Commit Message"\\
	
	\begin{block}{}	
		git commit
	\end{block}
 	If you don't specify -m "whatever", git will open a text editor, and you can type your commit message in there 
	Now the file is committed to your local repo, but not on the remote repo \emph{yet}.
\end{frame}

\begin{frame}
	\frametitle{Remote repositories}

	You now have a commit in your local repository that isn't in the remote repository.
	You "push" this to the remote using
	\begin{block}{}
		git push origin master
	\end{block}
	
	origin is the name that git gives to the place you cloned a repository from.\\
	master is a branch, I will explain that later.

\end{frame}

\begin{frame}
	\frametitle{A tutorial introduction}

	\begin{block}{Workflow}
		Your local repository consists of three "trees" maintained by git. The first one is 
		your \emph{Working Directory} which holds the actual files. the second one is the \emph{Index} which acts as a staging area and finally the \emph{HEAD} which points to the last commit you've made.
	\end{block}
	\begin{center}
		\includegraphics[scale=0.3]{trees.png}
	\end{center}
\end{frame}


\end{document}
